\documentclass[12pt]{article}
\usepackage{setspace}
\doublespacing
\usepackage[utf8]{inputenc}
\usepackage [english]{babel}
\usepackage [autostyle, english = american]{csquotes}
\MakeOuterQuote{"}
\usepackage{amssymb}
\usepackage{amsmath} %math package, for things like align%
\usepackage{graphicx}
\graphicspath{{../Figures/}}
\usepackage{listings}
\usepackage{empheq}
\usepackage{mathtools}
\usepackage[mathscr]{euscript}
\usepackage{indentfirst}
\usepackage[bottom]{footmisc}
\usepackage{wrapfig}
\usepackage[framed, numbered]{matlab-prettifier}
\usepackage{subcaption}
\usepackage{dsfont}
\usepackage{braket}
\renewcommand{\arraystretch}{.8}
\usepackage{multirow}
\usepackage[tmargin=1in,bmargin=1in,lmargin=1in,rmargin=1in]{geometry}
\usepackage[labelsep=period]{caption}
\captionsetup[table]{name=Table}
\renewcommand{\thetable}{\Roman{table}}
\newcommand*\widefbox[1]{\fbox{\hspace{2em}#1\hspace{2em}}}
\usepackage[numbered,framed]{matlab-prettifier} % For MATLAB code
\lstset{
	style              = Matlab-editor,
	basicstyle         = \mlttfamily,
	escapechar         = ",
	mlshowsectionrules = true,
}

\usepackage{titlesec}
% from the titlesec package
%\titleformat{ command }
%             [ shape ]
%             { format }{ label }{ sep }{ before-code }[ after-code ]
% \section*

 \captionsetup[table]{
	labelsep = newline,
	name = Table, 
	justification=justified,
	singlelinecheck=false,%%%%%%% a single line is centered by default
	labelsep=colon,%%%%%%
	skip = \medskipamount}
\begin{document}
	
%%%%%%%%%%%%%%%%%%%%%%%
\section*{Problem Setup}
Continuous time CWH Equations:
\begin{align*}
\ddot{x} &= 3\omega^2 x + 2\omega \dot{y} + u_1\\
\ddot{y} &= -2\omega \dot{x} + u_2,
\end{align*}
where $\omega \coloneqq \sqrt{\mu/R_0^3}$. For a spacecraft in LEO, I used $R_0 = R_e + 415$ km as the orbital radius, where $R_e = 6378.1$ km is the radius of the Earth. To write this system in state space form, let $x \coloneqq [x,\dot{x},y,\dot{y}]^\intercal\in\mathbb{R}^4$, so that we have the LTI system $\dot{x} = Ax + Bu$, where 
\begin{equation*}
A = 
\begin{bmatrix}
0 & 1 & 0 & 0\\
3\omega^2 & 0 & 2\omega & 0\\
0 & 0 & 0 & 1\\
0 & -2\omega & 0 & 0
\end{bmatrix}, \qquad
B = 
\begin{bmatrix}
0 & 0\\
1 & 0\\
0 & 0\\
0 & 1
\end{bmatrix}
\end{equation*}
To discretize the system, let $N = 20$ time steps with a time step $\Delta t = 0.5$, so that $T = N\Delta t = 10$. Assume a zero-order holder (ZOH) on the control and first order approximation, i.e. $A_d = I_4 + \Delta t A$ and $B_d = \Delta t B$. Adding noise to the system gives $x_{k+1} = A_d x_{k} + B_d u_k + G w_k$, where 
\begin{equation*}
A_d = 
\begin{bmatrix}
1 & \Delta t & 0 & 0\\
3\Delta t\omega^2 & 1 & 2\Delta t\omega & 0\\
0 & 0 & 1 & 1\\
0 & -2\Delta t\omega & 0 & 1
\end{bmatrix}, \qquad
B_d = 
\begin{bmatrix}
0 & 0\\
\Delta t & 0\\
0 & 0\\
0 & \Delta t
\end{bmatrix}, \qquad 
G = \text{diag}(10^{-4},10^{-4},5\cdot 10^{-8},5\cdot 10^{-8})
\end{equation*}
Boundary Conditions:
\begin{align*}
x_0 = 
\begin{bmatrix}
-1.5\\
-1.5\\
0.1\\
0.1
\end{bmatrix},\
x_f = 
\begin{bmatrix}
0\\
0\\
0\\
0
\end{bmatrix},\
\Sigma_0 = 10^{-2}\cdot\text{diag}(0.1,0.1,0.01,0.01),\ \Sigma_f = 0.5\Sigma_0
\end{align*}
The chance constraints are formulated as $\mathcal{X} = \bigcap_{j=1}^{2}\{x:\alpha_j^\intercal x \leq \beta_j\}$, where 
\begin{equation*}
\alpha_1 = 
\begin{bmatrix}
1\\
-1\\
0\\
0
\end{bmatrix},\
\alpha_2 = 
\begin{bmatrix}
1\\
1\\
0\\
0
\end{bmatrix},\
\beta_1 = \beta_2 = 0.5
\end{equation*}
This corresponds to a constraint space of a triangle, given by the intersection of $x - y \leq 0.5$ and $x + y \leq 0.5$. Lastly, the cost is 
\begin{equation*}
J = \sum_{k=0}^{N-1} x_k^\intercal Q x_k + u_k^\intercal R u_k,
\end{equation*}
where 
\begin{equation*}
Q = \text{diag}(10,10,1,1),\ R = \text{diag}(10^3,10^3)
\end{equation*}
This is the end \cite{b1}.

\bibliography{test} 
\bibliographystyle{IEEEtran}
%%%%%%%%%%%%%%%%%%%%%%%
\end{document}